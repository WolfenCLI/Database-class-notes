Il tuning delle interrogazioni ha solo benefici siccome, al contrario di quelli precedenti, non modifica nessuno schema ma ottimizza "manualmente" le interrogazioni.
\begin{itemize}
    \item Se ci sono espressioni spesso non vengono usati indici (riscrivere pre-calcolando le espressioni)
    \item In caso di Join
    \begin{itemize}
        \item Attributi numerici sono più veloci di attributi di tipo stringa
        \item Indici clusterizzati permettono la valutazione dell'operatore di merge join
    \end{itemize}
    \item Evitare \code{HAVING} quando possibile
    \item Utilizzare il meno possibile viste (non materializzate)
    \item Disgiunzioni di condizioni (\code{OR}) sostituite con \code{UNION}
    \item Non usare \code{DISTINCT} se non strettamente necessario
\end{itemize}

\subsection{Sottointerrogazioni}
\begin{itemize}
    \item Se \`e possibile, vengono rimosse
    \item Se sono scalari, viene calcolato il risultato una volta
    \item Se sono correlate devono invece essere eseguite ogni volta
\end{itemize}