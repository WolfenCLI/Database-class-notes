I trigger vengono utilizzati per eseguire automaticamente operazioni quando si verifica una determinata situazione.\\
I trigger si basano sul paradigma "evento-condizione-azione". \vspace{2mm} \\
\code{CREATE TRIGGER <nome trigger>\\
\{BEFORE|AFTER\} <evento> ON <tabella>\\
\ [REFERENCING \{OLD | NEW\} \{ROW | TABLE\} AS <variabile>]\\
\ [FOR EACH \{ROW | STATEMENT\}]\\
\ [WHEN <condizione>]\\
\{ <SQL> | BEGIN ATOMIC <Sequenza di comandi SQL> END \};}
\vspace{4mm} \\ \noindent
Possibili eventi:
\begin{itemize}
    \item \code{INSERT}
    \item \code{DELETE}
    \item \code{UPDATE [OF <lista attributi>]}
\end{itemize}
\vspace{4mm} \noindent
Modalità di esecuzione:
\begin{itemize}
    \item \textbf{Orientata all'istanza}: \code{FOR EACH ROW} eseguito per ogni tupla
    \item \textbf{Orientata all'insieme}: \code{FOR EACH STATEMENT} eseguito una volta per tutte le tuple
\end{itemize}

\begin{table}[htbp]
    \centering
    \begin{tabularx}{\textwidth}{| X | X | X |}
        \hline
        Modalità              & Evento                                        & Tabelle/Tuple di transizione \\ \hline
        \code{BEFORE ROW}     & \code{INSERT \newline DELETE \newline UPDATE} & \code{NEW \newline OLD \newline NEW, OLD} \\ \hline
        \code{AFTER ROW}      & \code{INSERT \newline DELETE \newline UPDATE} & \code{NEW, NEW TABLE \newline OLD, OLD TABLE \newline TUTTE} \\ \hline
        \code{BEFORE STATEMENT}      & \code{INSERT \newline DELETE \newline UPDATE} & \code{- \newline - \newline -} \\ \hline
        \code{AFTER STATEMENT}      & \code{INSERT \newline DELETE \newline UPDATE} & \code{NEW TABLE \newline OLD TABLE \newline NEW TABLE, OLD TABLE} \\ \hline
    \end{tabularx}
\end{table}
\vspace{4mm} \break \noindent
Ordine di selezione dei trigger:
\begin{enumerate}
    \item \code{BEFORE, FOR EACH STATEMENT}
    \item \code{BEFORE, FOR EACH ROW}
    \item \code{AFTER, FOR EACH ROW}
    \begin{itemize}
        \item Verifica dei vincoli
    \end{itemize}
    \item \code{AFTER, FOR EACH STATEMENT}
\end{enumerate}