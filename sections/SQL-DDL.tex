\subsection{Operazioni su Tabelle}
\subsubsection{Creazione}
Una tabella si genera con \code{CREATE TABLE <nome> (<nome colonna> <tipo>, ...)}.

\subsubsection{Vincoli e proprietà}
A destra del tipo si possono aggiungere alcune keywords per otterene dei \textbf{vincoli} o specifiche:
\begin{itemize}
    \item \code{DEFAULT <valore>}: Nel caso questa colonna non sia specificata nel relativo \code{INSERT} viene inserito il valore specificato
    \item \code{NOT NULL}: La colonna non può contenere valori nulli (di default può)
\end{itemize}
Dopo la dichiarazione delle colonne si includono i vincoli di \textbf{chiave}, \textbf{chiave primaria} e \textbf{chiave esterna}
\begin{itemize}
    \item \code{PRIMARY KEY (<colonna>[, <colonna>]*)}: Chiave primaria
    \item \code{UNIQUE (<colonna>[, <colonna>]*)}: Chiave (non primaria ma deve rimanere non ripetuta)
    \item \code{FOREIGN KEY (<colonna>[, <colonna>]*) REFERENCES <tabella>}: Chiave esterna
\end{itemize}
Questi ultimi vincoli sono anche esprimibili "in-line" dopo la dichiarazione della colonna tipo: \code{nome VARCHAR(20) PRIMARY KEY}.\\
SQL per aiutare a mantenere l'integrità dei dati permette di aggiungere dopo a \code{REFERENCES <tabella>} una clausola che specifica l'azione da eseguire nel caso che la chiave riferita (\code{t}) venga cancellata o modificata:
\begin{itemize}
    \item \code{ON DELETE <azione>}:
    \begin{itemize}
        \item \code{NO ACTION}: Se violerebbe il vincono, non la cancella
        \item \code{CASCADE}: Se \code{t} esiste nella tabella che la riferisce, ogni tupla che ne fa riferimento viene anche cancellata.
        \item \code{SET NULL}: Se \code{t} esiste nella tabella che la riferisce, ogni tupla che ne fa riferimento viene messa a \code{NULL}.
        \item \code{SET DEFAULT}: Se \code{t} esiste nella tabella che la riferisce, ogni tupla che ne fa riferimento viene messa al valore di default.
    \end{itemize}
    \item \code{ON DELETE <azione>}:
    \begin{itemize}
        \item \code{NO ACTION}: Se violerebbe il vincono, non la cancella
        \item \code{CASCADE}: Se \code{t} esiste nella tabella che la riferisce, ogni tupla che ne fa riferimento viene anche modificata.
        \item \code{SET NULL}: Se \code{t} esiste nella tabella che la riferisce, ogni tupla che ne fa riferimento viene messa a \code{NULL}.
        \item \code{SET DEFAULT}: Se \code{t} esiste nella tabella che la riferisce, ogni tupla che ne fa riferimento viene messa al valore di default.
    \end{itemize}
\end{itemize}

\subsubsection{Cancellazione}
\code{DROP TABLE <nome tabella> [RESTRICT|CASCADE]}.\\
Con \code{RESTRICT} viene cancellata solo se non riferita da altre tabelle. \code{CASCADE} invece cancella anche tutti gli elementi che la riferivano.

\subsubsection{Modifica}
\code{ALTER TABLE <nome tabella> <azione>}\\
Possibili azioni:
\begin{itemize}
    \item \code{ADD [COLUMN] <nome colonna> <tipo>}
    \item \code{ALTER [COLUMN] <nome colonna> \{SET DEFAULT <valore default\\nuovo> | DROP DEFAULT\}}
    \item \code{DROP [COLUMN] <come colonna> \{RESTRICT | CASCADE\}}
    \item \code{ADD CONSTRAINT <specifica vincolo>}: (un vincolo può essere ad esempio \code{UNIQUE})
    \item \code{DROP CONSTRAINT <nome vincolo>}
\end{itemize}

\subsection{Interrogazioni}
\code{SELECT [DISTINCT] \{<colonne>|*\}\\FROM <tabella[, tabella]*>\\WHERE <condizione booleana>}

\subsubsection{Alcuni operatori particolari}
\begin{itemize}
    \item \code{x BETWEEN a AND b}: $a \leq x \leq b$
    \item \code{a IN (a, b, c, ...)}: ritorna true se $a$ \`e contenuto nell'insieme indicato dopo
    \item \code{LIKE 'pattern'}: pattern matching in sql
\end{itemize}

\subsection{Operazioni insiemistiche}
Tutte le operazione di seguito riportate \textbf{eliminano eventuali duplicati}.
\begin{itemize}
    \item \code{UNION}: \textbf{Unione}
    \item \code{INTERSECT}: \textbf{Intersezione}
    \item \code{EXCEPT}: \textbf{Sottrazione}
\end{itemize}

\subsection{Ordinamento dei risultati}
Dopo al \code{WHERE}:\\
\code{ORDER BY <nome colonna> [ASC|DESC][, <nome colonna> [ASC|DESC]]*}

\subsection{Join}
\begin{itemize}
    \item \code{<nome relazione> CROSS JOIN <nome relazione>}: Prodotto cartesiano
    \item \code{<nome relazione> JOIN <nome relazione> ON <predicato>}:\\Restano solo le tuple che soddisfano il predicato
    \item \code{<nome relazione> NATURAL JOIN <nome relazione>}: Join fra colonne con lo stesso nome
    \item \code{<nome relazione> JOIN <nome relazione>\\USING <lista nomi colonne>}: Natural Join ma con colonne elencate
\end{itemize}

\subsection{Outer Join}
Gli outer join mantengono anche le tuple senza "associazioni" assegnando ai valori non conosciuti il valore \code{NULL}
\begin{itemize}
    \item \code{FULL OUTER JOIN}: mantiene tutte le tuple del join + le tuple non del join con i \code{NULL}
    \item \code{LEFT}: aggiunge al join tutte le tuple della tabella di sinistra con valori null per le colonne della tabella di destra
    \item \code{RIGHT}: il contrario del \code{LEFT}
\end{itemize}

\subsection{Funzioni di gruppo}
Se usate senza la clausola \code{GROUP BY <colonna o elenco>} agiscono su tutta la tabella.
\begin{itemize}
    \item \code{COUNT([DISTINCT] <colonna|*>)}
    \item \code{MIN(<colonna>)}
    \item \code{AVG(<colonna>)}
    \item \code{MAX(<colonna>)}
    \item \code{SUM(<colonna>)}
\end{itemize}
Si possono solo usare nella \code{SELECT} o nella \code{HAVING}\\
Quando si usa \code{GROUP BY}, le colonne elencate sono le uniche, oltre alle funzioni di gruppo su qualunque colonna, che possono essere messe nel \code{SELECT} siccome viene elencata una sola tupla per ogni gruppo e le colonne per le quali NON si raggruppa avranno valori diversi all'interno dello stesso gruppo.\\
\code{HAVING} mette una condizione per la quale di sceglie il gruppo. (filtra i gruppi)
\subsubsection{Modello di esecuzione}
\begin{enumerate}
    \item Si applica la condizione del \code{WHERE}
    \item Si divide la tabella a seconda del \code{GROUP BY}
    \item Si applica la condizione nel \code{HAVING}
    \item Si calcolano le funzioni di gruppo
\end{enumerate}
\subsubsection{Valori nulli}
Con le operazioni aritmetiche se \`e presente un valore \code{NULL} allora il risultato \`e \code{NULL}.\\
Con le operazioni di gruppo, i \code{NULL} vengono ignorati e viene ritornato un valore. Nel caso una colonna sia vuota o contenesse solo \code{NULL}, allora anche le funzioni di gruppo ritornano \code{NULL}.

\subsection{Viste}
\code{CREATE VIEW <nome vista> [(<lista nomi colonne>)]\\
AS <interrogazione>\\
\ [WITH [{LOCAL | CASCADE}] CHECK OPTION];}\vspace{4mm} \\
Sulle viste si possono eseguire:
\begin{itemize}
    \item interrogazioni
    \begin{itemize}
        \item Restrizione: non è ammesso l’uso di funzioni di gruppo su colonne di viste che sono a loro volta definite tramite funzioni di gruppo
    \end{itemize}
    
    \item inserimenti
    \begin{itemize}
        \item la query di definizione Q soddisfa tutte le restrizioni richieste per il DELETE
        \item tutte le colonne di V soddisfano le restrizioni richieste per l’UPDATE
        \item tutte le colonne di R su cui vale il vincolo di NOT NULL e per cui non è specificato valore di default sono incluse in V
    \end{itemize}
    
    \item aggiornamenti
    \begin{itemize}
        \item la query di definizione Q soddisfa tutte le restrizioni richieste per il DELETE
        \item C non è definita tramite un’espressione o funzione
    \end{itemize}
    
    \item cancellazioni
    \begin{itemize}
        \item è su una singola relazione R
        \item non contiene
        \begin{itemize}
            \item GROUP BY
            \item DISTINCT
            \item operatori insiemistici
            \item funzioni di gruppo
        \end{itemize}
        \item eventuali sotto-interrogazioni in Q non fanno riferimento ad R
    \end{itemize}
\end{itemize}

\subsection{Sotto-interrogazioni}
Interrogazioni all'interno di un'altra interrogazione. Di solito nella clausola \code{WHERE}.
\begin{itemize}
    \item \textbf{Sotto-interrogazione semplice}: Ritorna un singolo valore (scalare) o più. Nel casi si abbia più di un valore, viene trattato come un insieme e serve quindi l'operatore \code{IN}
    \item \textbf{Sotto-interrogazione Table Subquery}: usano i quantificatori \code{ANY} (almeno una tupla della sotto query e \code{ALL} (tutte le tuple della sottoquery)
    \item \textbf{Sotto-interrogazione Correlata}: dipendono dalla tupla candidata
\end{itemize}

\subsection{Exists o Not Exists}
\code{EXISTS} e \code{NOT EXISTS} possono essere utilizzati rispettivamente al posto di \code{INTERSECT} e \code{EXCEPT} (sottrazione)

\subsection{Divisione}
La divisione viene implementata tramite l'uso di due \code{NOT EXISTS} in due sotto-query (\code{WHERE} non esiste film di "autore" dove non esiste alcun noleggio)\\
Inoltre si può anche esprimere tramite funzioni di gruppo con \code{HAVING COUNT(...) = (sotto-query con COUNT nel SELECT)}
